\input{config/config.tex}
\usepackage{ulem}

\fancyhead[L]{
Application No.: 13/763,627 \newline 
Amdt. dated : January 2, 2014 \newline
Reply to office action of October 4, 2013}

\begin{document}
IN THE UNITED STATES PATENT AND TRADEMARK OFFICE

\heading{I. CAPTION}

In re the Application of: PATENT \newline
Applicant: Vladimir Vassilev \newline
Application No.: 13/763,627 \newline
Application Title: Digital aerophones and dynamic impulse response
systems \newline
Examiner: Warren, David S \newline
Art Unit: 2832 \newline


\begingroup
\setlength\parindent{3.6in}
Sverderups gate 1A, 0559 Oslo \\ January 2, 2014
\endgroup

Commissioner for Patents \newline
Post Office Box 1450 \newline
Alexandria, Virginia 22313-1450 \newline

Sir:
\begin{center}
AMENDMENT
\end{center}


\heading{II. INTRODUCTION}

In response to the Official Action dated October 4, 2013, please amend
this application as follows:

\textbf{Amendment to the Claims} are reflected in the listing of claims which begins on page 2 of this paper.

\textbf{Remarks} begin on page 3 of this paper.

\newpage
\heading{III. AMENDMENTS}

This listing of claims will replace all prior versions, and listings, of
claims in the application:

\cl (currently amended): \sout{Aparatus} \uline{Apparatus} for dynamic estimation of the impulse response of \sout{dynamic acoustic systems} \uline{acoustic channels part of the volume of the acoustic system of the resonator of an aerophone instrument} comprising probing signal generator, transmiting transducer, receiving transducer and signal processing block.

\cl (currently amended): Method of \sout{human} \uline{human-computer} interaction comprising \sout{signal system generating the interaction signal with at least one subsystem of dynamic linear type with impulse response equal to the dynamically estimated impulse response of a dynamic acoustic system the human interacts with} \uline{: human interacting with the resonator of an aerophone instrument, dynamic probing of the impulse responses of acoustic channels part of the acoustic system of the resonator, signal processing algorithm determining the dynamic state of the interaction from the dynamic impulse response signals}.

\cl (currently amended): Method of playing aerophone instruments \sout{using dynamically estimated impulse response of the acoustic system of the resonator} \uline{, the method comprising: player manipulating the resonator of the instrument as he is used to play the original instrument, probe inserted into the acoustic system of the resonator continuously probing the impulse responses of acoustic channels part of the acoustic system of the resonator, signal processing algorithm determinig the waveform of the sound signal produced as output from the instrument based on the dynamic impulse responses}.

\cl (new): A method as in claim 2, wherein said the signal processing algorithm determining the dynamic state of the interaction choses its output state out of a predefined set of states using a least sum of squares of differences comparison with the predefined impulse responses corresponding to each of the states in the set.

\newpage
\begin{center}
REMARKS
\end{center}

Drawings

Quote from office action of October 4, 2013:

\begin{quote}
Figure 6 should be designated by a legend such as --Prior Art-- because only
that which is old is illustrated. See MPEP \S 608.02(g). Corrected drawings in
compliance with 37 CFR 1.121 (d) are required in reply to the Office action to avoid
abandonment of the application. The replacement sheet(s) should be labeled
"Replacement Sheet" in the page header (as per 37 CFR 1.84(c)) so as not to obstruct
any portion of the drawing figures. If the changes are not accepted by the examiner, the
applicant will be notified and informed of any required corrective action in the next Office
action. The objection to the drawings will not be held in abeyance.
\end{quote}

Replacement Sheet labeling Figure 6 as --Prior Art-- has been submitted.

Claim Rejections - 35 USC  \S 112

Quote from office action of October 4, 2013:

\begin{quote}
Claim 2 recites a
"method of human interaction comprising signal system generating the interaction
signal..." There does not appear to be any method step associated with human
interaction.
\end{quote}

Claim 2 has been amended to include the method step associated with human interaction "human interacting with the resonator of an aerophone instrument".

\begin{quote}
The Examiner cannot ascertain whether the Applicant intends to claim a
method of human interaction or a method of generating a signal.Furthermore, the
claim language is not clear, e.g., what is meant by "signal system generating
the...signal"?
\end{quote}

Claim 2 has been amended to clarify that the method claimed is for human-computer interaction and it is based on input from the human resulting in a signal which can be further processed by a computer.

\begin{quote}
There is no antecedent for "the interaction signal." There is no
antecedent for "the dynamically estimated impulse response." Correction is required.
\end{quote}

Claim 2 has been amended to provide antecedents for "the interaction signal." which is the product of the signal processing algorithm which processes the dynamic impulse response signals. Since Claim 2 is not constrained to a specific algorithm this can be any algorithm which can retrieve a useful state information from the dynamic impulse response signals. A new Claim 4 is added wherein the interaction signal is a digital signal with predefined number of states where the current state is the one that has the least sum of squares of difference when compared to the current impulse response.

Quote from office action of October 4, 2013:

\begin{quote}
Regarding claim 3, again, the Applicant's invention appears to be a method, but no
method steps are claimed.
\end{quote}

Claim 3 has been amended to clearly define the method steps.

\begin{quote}
Here the applicant appears to be claiming a method of
playing using an impulse response. As known by the Examiner, impulse response is a
measurement. How does one play an instrument with a measurement? Furthermore,
the Examiner cannot ascertain whether the "aerophone," the "acoustic system" and the
"resonator" are the same element or different. Clarification and/or correction required.
\end{quote}

Claim 3 has been amended to add a missing step related to the digital signal processing algorithm which uses the dynamic impulse response signals synthesizing a sound waveform signal. Claim 3 is not constrained to a specific algorithm. The Specification collaborates on a example using such algorithm based on the least sum of squares of difference selecting a tone from a predefined set of known states and generating a waveform corresponding to that known state.

Claim Rejections - 35 USC  102

Quote from office action of October 4, 2013:

\begin{quote}
Claim 1 is rejected under pre-AIA 35 U.S.C. 102(b) as being anticipated by
O'Sullivan (2007/0237335). The Applicant appears to be claiming the generic elements
of virtually any impulse response measurement system. O'Sullivan shows a typical
impulse response system (see figs. 1 and 4), wherein a probing signal generator (driver
for element 10), transmitting transducer (10), receiving transducer (14), and a signal
processing block (22, see paragraph [O026]). These are the essential elements of
virtually any impulse response measuring system.
\end{quote}

Claim 1 has been amended to specifically define that the proposed apparatus is targeted at probing the dynamic impulse response of acoustic channels part of the volume of the acoustic system of the resonator of an aerophone instrument. The innovative idea is that the aerophone instruments in particular are very well suited for use with such an apparatus.

Quote from office action of October 4, 2013:

\begin{quote}
Claim 1 is rejected under pre-AIA 35 U.S.C. 102(b) as being anticipated by Ohta
(2003/0159569). See fig. 1.
\end{quote}

As mentioned above the innovation value of Claim 1 is the very specific application of a known method. With the current amendment this is made clear. The transformation of the aerophone model from the prior art model shown in FIG.6 to the model in FIG.7 makes certain compromises explained in the Specification provides grounds for considering the apparatus claimed in Claim 1 an invention.

Quote from office action of October 4, 2013:

\begin{quote}
Claim 2 is rejected under pre-AIA 35 U.S.C. 102(b) as being anticipated by Ryle
(2007/0227344). As best as can be determined (see 112 rejection supra), Ryle
discloses a method of human interaction (guitar playing) comprising generating an
impulse response signal with estimated response of the system the user interacts with
\end{quote}

Ryle (2007/0227344) uses static impulse responses as part of FIR filters used in a signal system transforming the signal from a guitar pickup. I do not agree this patent has any grounds for rejecting any of the claims in the invention which is based on dynamic acoustic system identification using active probing and the modeling of aerophone instruments. A much more relevant area would be the sonar and ultrasound imaging which heavily uses linear acoustic channel system identification. However the particular field of application is aerophones and I have not found any reason to believe there is any prior art for rejection under U.S.C. 102(b).

In view of the above, it is submitted that the claims are in a condition for allowance. Reconsideration of the objections and rejections is requested.

\setlength\parindent{3.6in}
Respectfully submitted, \\Vladimir Vassilev
\end{document}

